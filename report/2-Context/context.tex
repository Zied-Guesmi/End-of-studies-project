%!TEX root = ../report.tex
%*******************************************************************************
%                                   Context                                    %
%*******************************************************************************

\chapter{Project context}

%******************************** Section 1.1 *********************************%
\section{Introduction}
  The aim of this chapter is to contextualize the project by giving its background,
  to present the host company and to define in a later section the main problem as well
  as the way we adresse it.

%******************************** Section 1.2 *********************************%
\section{Project background}
  This project is about desining and implementing a positive energy worker.
  A Raspberry Pi based system that powers iExec's infrastructure and serves - at the same
  time - as an IoT device to embrace the Edge computing concept.
  It is achieved in the context of the preparation of the end of studies project submitted to
  obtain computer science engineering degree.

%******************************** Section 1.3 *********************************%
\section{Host company}

  \begin{figure}[!h]\centering
    \includegraphics[width=.5\columnwidth]{2-Context/figs/iExec-logo.pdf}
    \caption{iExec logo}
  \end{figure}

  Started in 2016, iExec\cite{iExec} was co-founded by Dr. Gilles Fedak and Pr. Haiwu He.

  \subsection{Presentation of the founders}
    Ph.D, CEO and Co-Founder, Dr. Gilles Fedak has been a permanent INRIA research scientist since
    2004 at the ENS in Lyon, France. His research interests lie in Parallel and Distributed
    Computing, with a particular emphasis on the problematic of using large and loosely-coupled
    distributed computing infrastructures to support highly demanding computational and
    data-intensive science. He co-authored about 80 peer-reviewed scientific papers and won two
    Best Paper awards.
  
    Pr. Haiwu He, Ph.D, Co-Founder and Head of Asian-Pacific Region was a
    research engineer expert at INRIA Rhone-Alpesin Lyon, France from 2008 to 2014. He has
    published about 30 refereed journal and conference papers. His research interest covers
    peer-to-peer distributed systems, cloud computing, and big data.
  
  \subsection{Mission of the company}
    iExec aims at providing decentralized applications running on the blockchain a scalable,
    secure and easy access to the services, data-sets and computing resources they need. This
    technology relies on Ethereum smart contracts and allows the building of a virtual cloud
    infrastructure that provides high-performance computing services on demand.

    iExec leverages a set of research technologies that have been developed at the INRIA and CNRS
    research institutes in the field of Desktop Grid computing. The idea of Desktop Grid
    (aka. Volunteer Computing) is to collect the computer resources that are underutilized on the
    Internet to execute very large parallel applications at the fraction of the cost of a
    traditional supercomputer. iExec relies on XtremWeb-HEP\cite{xtremweb}, a mature, solid, and open-source Desktop
    Grid software which implements all the needed features: fault-tolerance, multi-applications,
    multi-users, hybrid public/private infrastructure, deployment of virtual images, data management,
    security and accountability, and many more.

    iExec is developing a new Proof-of-Contribution\cite{POCO} (PoCo) protocol, that will allow off-chain
    consensus. Thanks to the Proof-of-Contribution, external resource providers will have the usage
    of their resources certified directly in the blockchain.

    iExec aims to deploy a scalable, high-performance, secure and manageable infrastructure sidechain
    that will promote a new form of distributed governance, involving key HPC, big data and cloud
    industry leaders.

  \subsection{Company's crypto-currency}
    iExec is using its ERC20-compliant token to provide standard and secure payments. RLC\cite{RLC} which stands for
    "Run on Lots of Computers", can be securely and easily stored, transferred, traded, divided and used
    to make payments. This widely adopted cryptocurrency (87 million RLC are currently in circulation)
    is used to access all iExec's services.

  \subsection{Company's roadmap}

  \begin{itemize}
    
    \item \textit{05/01/16 - Project Launch: }
    Haiwu He and Gilles Fedak announce iExec at the Blockchain Show in Beijing.

    \item\textit{09/09/16 - Devcon2 Presentation: }
    The project is presented in Shanghai to the Ethereum community at the Devcon2.

    \item\textit{10/17/16 - Company Creation: }
    iExec is founded as a French company.

    \item\textit{11/12/16 - Tsinghua University Acceleration Program: }
    iExec is accepted into the Tsinghua University SEM X-elerator, one of the most famous startup accelerators in China.

    \item\textit{11/13/16 - Super Computing 2016: }
    iExec participates to SC’16, the International Conference for High Performance Computing, Networking, Storage and Analysis

    \item\textit{11/01/16 - Whitepaper Release: }    
    The iExec Whitepaper is released. The document presents the technological foundations of the project, its roadmap and details about the RLC token crowdsale.

    \item\textit{02/17/17 - PoC I Release: }
    Vanity Gen is the first application to run on the iExec PoC. It requires a lot of computing power to get the first character of a personalized Bitcoin address.

    \item\textit{04/19/17 - Token Sale: }
    iExec raises 2,762 BTC and 17,388 ETH in less than 3 hours, which represents the equivalent of 12 million USD. It was the 5th largest ICO at the time.

    % \item\textit{04/20/17 - Bittrex Listing: }
    % The Bittrex exchange platform opens an RLC/BTC market.

    \item\textit{05/30/17 - PoC II Release: }
    Stockfish, a famous open-source chess engine is the second application deployed on the iExec PoC.

    \item\textit{07/11/17 - iExec Joins EEA: }
    iExec joins the Ethereum Enterprise Alliance to augment Ethereum in terms of on privacy, confidentiality, scalability, and security.

    \item\textit{11/04/17 - SDK 1.0 Release: }
    The release of the iExec V1 (aka « The Wanderer ») is announced during Devcon3 in Cancún, Mexico.

    \item\textit{09/26/17 - Explorer Release: }
    The iExec Explorer allows anyone to see all the transactions and application registrations within the iExec network.

    \item\textit{12/01/17 - DApp Challenge Launch: }
    The DApp Challenge is announced to boost the development of decentralized apps, with a prize pool of \$150,000.

    \item\textit{12/15/17 - Hong Kong Office: }
    iExec opens a new office in Hong Kong.

    \item\textit{12/20/17 - DApp Store Release: }
    The DApp Store lists all the applications built on top of iExec.

    \item\textit{12/21/17 - Mainnet Deployment: }
    The team deploys the iExec Smart Contracts on the Ethereum Mainnet for the first time.

    \item\textit{01/09/18 - iExec Joins the OpenFog Consortium: }
    iExec joins the OpenFog Consortium to accelerate the deployment of fog computing technologies.

    \item\textit{01/12/18 - Binance Listing: }
    Binance, one of the most popular exchange platforms, lists RLC.

    \item\textit{01/24/18 - The First Cloud Computing Provider Joins iExec: }
    Stimergy becomes the first-ever cloud computing provider to join the iExec marketplace.

    \item\textit{01/31/18 - The First Invoice in RLC: }
    iExec’s new partner, Cloud\&Heat, issues the first invoice ever to be paid in RLC.

    \item\textit{02/09/18 - Docker Support: }
    Developers can dockerize their apps and run them on the iExec cloud.

    % \item\textit{12/21/17 - Mainnet Deployment: }
    % Blender Deployment
    % Blender, a CGI rendering app, is deployed on top of iExec and becomes the first use case of 3D rendering on iExec.
    % 02/15/18

    \item\textit{05/31/18 - iExec 2.0 — Cloud Marketplace: }
    The version 2 will include the full marketplace platform network, with the PoCo algorithm enabling the first decentralized cloud.

    \item\textit{05/31/19 - iExec 3.0 — Hybrid Public/Private Infrastructure: }
    The version 3 includes key features for the enterprises to widely adopt the iExec market network by providing them with full control over the private/ public employment of their resources.

    \item\textit{05/31/20 - Version 4.0 — High Performance Computing: }
    This version allows miners to join the iExec market network providing true supercomputing capabilities.

    \item\textit{05/31/21 - iExec 5.0 — Beyond the Distributed Cloud: }
    The goal of this version is to allow new usage of iExec beyond the decentralized Cloud: FoG/Edge computing, ambiant AI, IoT+ Big Data, smart cities, VR, and more.

  \end{itemize}

  From a research project, iExec is now a company, whose headquarters are in Lyon, France, with a
  subsidiary in Hong Kong.

  \subsection{Presentation of the supervisor}

%******************************** Section 1.4 *********************************%
\section{Problematic}
  With the huge increase of human dependency on Information Technology for even daily activities,
  comes the massive consumption of electricity. In 2016, global data centers used roughly 416
  terawatts (more than 90 billion kilowatt-hours for just U.S. data centers) or about 3\% of the
  total electricity in terms of percentage, which is nearly 40\% more than the entire United Kingdom.
  Predictions have shown that this consumption will double every four years\cite{consumption-prediction}.

  As processing-power demanding technologies like artificial intelligence and blockchain have
  been appearing, the network of data centers that have sprung up in the past decade will spread.
  Moreover, internet-connected devices is changing the entire landscape because IoT is
  projected to exceed 20 billion devices by 2020. Given there are currently 10 billion devices,
  doubling that will require huge increases to our data center infrastructure, which will
  massively grow our electricity consumption, and that is just adding fuel to the fire.

  There has been some trials to remedy the situation such as using other alternatives to silicon in
  data storage and counting on virtualisation to reduce the use of physical machines, but all of that
  still can not keep up against the extreme consumption demand.

%******************************** Section 1.5 *********************************%
\section{Suggested solution}
  Although the existing trials are aiming to make datacenters greener, in this project we are taking another
  perspective. The idea is to push computation (or part of it) to the edge of the network instead of sending
  it to the cloud (fog/edge computing principles) and use solar energy to power devices that execute this
  computation. iExec plays a key role in this approach because the execution process will be handled by its
  software, in other words we are creating an iExec's worker that is completely autonomous and energy positive.
  The prototype of this system is a Raspberry Pi based device powered by a solar panel but the concept can be
  applied to a wider range of use cases.

%******************************** Section 1.6 *********************************%
\section{Conclusion}
  After talking about the project's context and background, we presented the host company and discussed the
  problematic we are dealing with as well as the solution to resolve this issue, which is the subject of this
  project.

\clearpage