Fiche de synthèse
Dans le cadre de la gestion informatisée des stages de PFE et de l’archivage des rapports de PFE, nous vous demandons de renseigner les items suivants : 
Formation (LFI3 / IF5) :
Année Universitaire : 
Session (Principale / Juillet / Septembre) :
Auteur(s)  (Nom, prénom) : 


Titre du rapport :

Titre traduit en français (pour les titres en langue étrangère) : 

Organisme d’accueil : 
Pays d’accueil : 
Responsable de stage (nom et prénom) : 
Email du Responsable de stage : 
Tél. du Responsable de stage :
Mots-clés caractérisant votre rapport (4 à 5 mots maximum) : 

Charte de non plagiat

\clearpage

Protection de la propriété intellectuelle

Tout travail universitaire doit être réalisé dans le respect intégral de la propriété intellectuelle d’autrui. Pour tout travail personnel, ou collectif, pour lequel le candidat est autorisé à utiliser des documents (textes, images, musiques, films etc.), celui-ci devra très précisément signaler le crédit (référence complète du texte cité, de l’image ou de la bande-son utilisés, sources internet incluses) à la fois dans le corps du texte et dans la bibliographie. Il est précisé que l’UCO dispose d’un logiciel anti-plagiat, aussi est-il demandé à tout étudiant de remettre à ses enseignants un double de ses travaux lourds sur support informatique.
Cf. « Prévention des fraudes à l’attention des étudiants »


Je soussigné(e), ……………………………………………………., étudiant(e) en …………………………………………………     m’engage à respecter cette charte.


Fait à ……………………………………..……………, le……………………………………..


Signature : 
