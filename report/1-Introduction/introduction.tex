%!TEX root = ../report.tex
%*******************************************************************************
%                            General Intro                                     %
%*******************************************************************************

\chapter*{General Introduction} \addcontentsline{toc} {chapter} {General Introduction}
% \thispagestyle{plain}

% \thispagestyle{fancy}
% \fancyhf{}
% \rhead{Share\LaTeX}
% \lhead{Guides and tutorials}
% \rfoot{Page \thepage}

Whether we are online shopping, streaming videos or reading the feed on social media, every
internet activity involves huge amounts of data that needs to be stored and processed
somewhere. Data centers are there all over the word to accomplish this mission. They have
already spread from virtually nothing 10 years ago to consuming about 3 per cent of the
global electricity supply and accounting for about 2 per cent of total greenhouse gas
emissions. That gives it the same carbon footprint as the airline industry\cite{consumption-stats}.

\vspace{1cm}
\say{
    \textit{If we carry on going the way we have been it would become unsustainable – this level
    of data centre growth is not sustainable beyond the next 10 to 15 years. The question is,
    what are we going to do about it?}
}

\begin{flushright}
    \textit{Professor Ian Bitterlin}
    \footnote{
        Professor Ian Bitterlin is a Chartered Engineer with more than 25 years’ experience
        in data-center power and cooling, CTO of Emerson Network Power Systems in EMEA
        and a Visiting Professor at University of Leeds in the School of Mechanical Engineering.
    }
\end{flushright}
\vspace{0.5cm}

We have already improved everything around compute architecture, the only thing left to improve
is the compute side of the equation. It’s either a breakthrough in our perspective about it, or we
need to get deadly serious about doubling the number of power plants on the planet.
One way to curb their carbon footprint is to increase the amount of renewable energy they
use. But even if the industry was able to shift to 100 per cent renewable electricity, the volume
of energy they would need would put intolerable pressure on the world’s power systems.
So which way the industry decides to go could have a huge bearing on whether renewable energy
receives the huge investment that drives innovation – bringing down the cost of green electricity
to everybody’s benefit. It could also play a large role in determining whether the world can
avoid the worst ravages of global warming.

In this context, this project, originally suggested by iExec, is performed as part of the
preparation to obtain the computer science engineering degree from the Faculty of Mathematical,
Physical and Natural Sciences of Tunis, University of Tunis ELMANAR, Tunisia. It aims to suggest a
solution to this problem in an innovative approach by embracing Edge computing concept combined
with solar energy. The project takes place in the premises of the company in Lyon, France for a
duration of 6 months starting from March 2018. The work has been done under the supervision of
Mr. Heithem Abbess from the university and Mr. Ugo Plouvier from the company.

This report
illustrates the work that has been done, from design to requirements and implementation. Those topics
are covered in five chapters, we start by contextualizing the project and discussing the state of the
art with a comparison between our perspective and some existing ones. Then, we present the different
requirements and technical specifications which leads us to the implementation details. Finally we
conclude and suggest some improvements to the current limitations.
