%!TEX root = ../thesis.tex
%*******************************************************************************
%                                   Context                                    %
%*******************************************************************************

\chapter{Project Context}

%******************************** Section 1.1 *********************************%
\section{Introduction}
  The aim of this chapter is to contextualize the project by giving its background,
  to present the host company and to define in a later section the main problem as well
  as the way we adresse it.

%******************************** Section 1.2 *********************************%
\section{Project Background}
  This project is about desining and implementing a positive energy worker.
  A Raspberry Pi based system that powers iExec's infrustructure and serves - at the same
  time - as an IoT device to embraces the Edge computing concept.
  It is achieved in the context of the preparation of the end of studies project submitted to
  obtain computer science engineering degree.

%******************************** Section 1.3 *********************************%
\section{Host company}

  \begin{figure}[!h]\centering
    \includegraphics[width=.5\columnwidth]{2.0-Context/figs/iExec-logo.pdf}
    \caption{iExec's logo}
    % \label{iExec's logo}
  \end{figure}

  Started in 2016, iExec\cite{iExec} was co-founded by Dr. Gilles Fedak and Pr. Haiwu He.\\
  Ph.D, CEO and Co-Founder, Dr. Gilles Fedak has been a permanent INRIA research scientist since
  2004 at the ENS in Lyon, France. His research interests lie in Parallel and Distributed
  Computing, with a particular emphasis on the problematic of using large and loosely-coupled
  distributed computing infrastructures to support highly demanding computational and
  data-intensive science. He co-authored about 80 peer-reviewed scientific papers and won two
  Best Paper awards. Pr. Haiwu He, Ph.D, Co-Founder and Head of Asian-Pacific Region was a
  research engineer expert at INRIA Rhone-Alpesin Lyon, France from 2008 to 2014. He has
  published about 30 refereed journal and conference papers. His research interest covers
  peer-to-peer distributed systems, cloud computing, and big data.

  iExec aims at providing decentralized applications running on the blockchain a scalable,
  secure and easy access to the services, data-sets and computing resources they need. This
  technology relies on Ethereum smart contracts and allows the building of a virtual cloud
  infrastructure that provides high-performance computing services on demand.

  iExec leverages a set of research technologies that have been developed at the INRIA and CNRS
  research institutes in the field of Desktop Grid computing. The idea of Desktop Grid
  (aka. Volunteer Computing) is to collect the computer resources that are underutilized on the
  Internet to execute very large parallel applications at the fraction of the cost of a
  traditional supercomputer. iExec relies on XtremWeb-HEP\cite{xtremweb}, a mature, solid, and open-source Desktop
  Grid software which implements all the needed features: fault-tolerance, multi-applications,
  multi-users, hybrid public/private infrastructure, deployment of virtual images, data management,
  security and accountability, and many more.

  iExec is developing a new Proof-of-Contribution\cite{POCO} (PoCo) protocol, that will allow off-chain
  consensus. Thanks to the Proof-of-Contribution, external resource providers will have the usage
  of their resources certified directly in the blockchain.

  iExec aims to deploy a scalable, high-performance, secure and manageable infrastructure sidechain
  that will promote a new form of distributed governance, involving key HPC, big data and cloud
  industry leaders.

  iExec is using its ERC20-compliant token to provide standard and secure payments. RLC\cite{RLC} which stands for
  "Run on Lots of Computers", can be securely and easily stored, transferred, traded, divided and used
  to make payments. This widely adopted cryptocurrency (87 million RLC are currently in circulation)
  is used to access all iExec's services.

  From a research project, iExec is now a company, whose headquarters are in Lyon, France, with a
  subsidiary in Hong Kong.

%******************************** Section 1.4 *********************************%
\section{Problematic}
  With the huge increase of human dependency on Information Technology for even daily activities,
  comes the massive consumption of electricity. In 2016, global data centers used roughly 416
  terawatts (more than 90 billion kilowatt-hours for just U.S. data centers) or about 3\% of the
  total electricity in terms of percentage, which is nearly 40\% more than the entire United Kingdom.
  Predictions have shown that this consumption will double every four years\cite{consumption-prediction}.

  As processing-power demanding technologies like artificial intelligence and blockchain have
  been appearing, the network of data centers that have sprung up in the past decade will spread.
  Moreover, internet-connected devices is changing the entire landscape because IoT is
  projected to exceed 20 billion devices by 2020. Given there are currently 10 billion devices,
  doubling that will require huge increases to our data center infrastructure, which will
  massively grow our electricity consumption, and that is just adding fuel to the fire.

  There has been some trials to remedy the situation such as using other alternatives to silicon in
  data storage and counting on virtualisation to reduce the use of physical machines, but all of that
  still can not keep up against the extreme consumption demand.

%******************************** Section 1.5 *********************************%
\section{Suggested Solution}
  Although the existing trials are aiming to make datacenters greener, in this project we are taking another
  perspective. The idea is to push computation (or part of it) to the edge of the network instead of sending
  it to the cloud (fog/edge computing principles) and use solar energy to power devices that execute this
  computation. iExec plays a key role in this approach because the execution process will be handled by its
  software, in other words we are creating an iExec's worker that is completely autonomous and energy positive.
  The prototype of this system is a Raspberry Pi based device powered by a solar panel but the concept can be
  applied to a wider range of use cases.

\section{Conclution}
  After talking about the project's context and background, we presented the host company and discussed the
  problematic we are dealing with as well as the solution to resolve this issue, which is the subject of this
  project.

\clearpage