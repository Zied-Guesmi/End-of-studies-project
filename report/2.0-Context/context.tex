%!TEX root = ../thesis.tex
%*******************************************************************************
%                                   Context                                    %
%*******************************************************************************

\chapter{Project Context}

\ifpdf
    \graphicspath{{2.0-Context/Figs/Raster/}{2.0-Context/Figs/PDF/}{2.0-Context/Figs/}}
\else
    \graphicspath{{2.0-Context/Figs/Vector/}{2.0-Context/Figs/}}
\fi

\section{Context}
- pfe \\
- Engineering degree

\section{Host company}
Started in 2016, iExec was co-founded by Dr. Gilles Fedak and Pr. Haiwu He.

Ph.D, CEO and Co-Founder, Dr. Gilles Fedak has been a permanent INRIA research scientist since 2004
at the ENS in Lyon, France. His research interests lie in Parallel and Distributed Computing,
with a particular emphasis on the problematic of using large and loosely-coupled distributed
computing infrastructures to support highly demanding computational and data-intensive science.
He co-authored about 80 peer-reviewed scientific papers and won two Best Paper awards.
Pr. Haiwu He, Ph.D, Co-Founder and Head of Asian-Pacific Region was a research engineer expert at
INRIA Rhone-Alpesin Lyon, France from 2008 to 2014. He has published about 30 refereed journal and
conference papers. His research interest covers peer-to-peer distributed systems, cloud computing,
and big data.

iExec aims at providing decentralized applications running on the blockchain a scalable, secure
and easy access to the services, data-sets and computing resources they need. This technology
relies on Ethereum smart contracts and allows the building of a virtual cloud infrastructure that
provides high-performance computing services on demand.

iExec leverages a set of research technologies that have been developed at the INRIA and CNRS
research institutes in the field of Desktop Grid computing. The idea of Desktop Grid (aka. Volunteer
Computing) is to collect the computer resources that are underutilized on the Internet to execute
very large parallel applications at the fraction of the cost of a traditional supercomputer.
iExec relies on XtremWeb-HEP, a mature, solid, and open-source Desktop Grid software which implements
all the needed features: fault-tolerance, multi-applications, multi-users, hybrid public/private
infrastructure, deployment of virtual images, data management, security and accountability,
and many more.

iExec is developing a new Proof-of-Contribution (PoCo) protocol, that will allow off-chain consensus.
Thanks to the Proof-of-Contribution, external resource providers will have the usage of their
resources certified directly in the blockchain.

iExec aims to deploy a scalable, high-performance, secure and manageable infrastructure sidechain
that will promote a new form of distributed governance, involving key HPC, big data and cloud industry
leaders.

iExec is using its ERC20-compliant token to provide standard and secure payments. RLC which stands for
"Run on Lots of Computers", can be securely and easily stored, transferred, traded, divided and used
to make payments. This widely adopted cryptocurrency (87 million RLC are currently in circulation)
is used to access all iExec's services.

From a research project, iExec is now a company, whose headquarters are in Lyon, France, with a
subsidiary in Hong Kong.

\section{Problematic}
- Enegy Consumption \\
- Centralized Services \\
- Idle IoT devices \\
- Idle Computing resources 

\section{Suggested Solution}
- Positive Energy Worker
- Usefull use cases
- Multi-functionality IoT devices

\section{Adopted Methodology}
- To Do

\clearpage