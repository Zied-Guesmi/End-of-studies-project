%!TEX root = ../report.tex
%*******************************************************************************
%                    General Conclusion And Perspectives                       %
%*******************************************************************************


\chapter*{General Conclusion And Perspectives}
\addcontentsline{toc} {chapter} {General Conclusion And Perspectives}

The IT industry is making a revolutionary transformation, but energy consumption is one of the main problems
that faces this field. Our project, realized in iExec company suggests an approach to resolve this issues and
leverages the importance of renewable energy in general and solar energy in particular. It is a part of
iExec's initiative to remedy the problem of energy consumptions of data centers by employing some key notions
such as Fog/Edge computing. Fortunately, there is more than enough available renewable energy to meet all of
our needs, if we can harness it.

Unfortunately, we're starting from a point at which less than 2\% of the world's energy comes
from renewables like wind, solar and geothermal. Our challenge then is to make that 2\% fraction grow to
replace about 86\% of the world's current primary energy, in 90 years or less.

The Positive Energy Worker (PEW), is a Raspberry Pi based machine, powered by a solar panel, that executes
received jobs in the iExec ecosystem, in addition to providing possible collected data. This project is
a prototype and proof of concept of the solution, so it can be extended and applied on a wider range of
use cases.

In a first step of our internship, we focused on learning about photovoltaic systems and understanding iExec's
ecosystem and tools. Then, we designed and implemented our requirements which led us finally to the testing
phase.

Although, the system is well functioning and almost autonomous, the software can be more optimized to improve
the performance of the worker despite its limited resources.

This project was a good opportunity to apply the knowledge acquired during the engineering studies in different
subjects such as system administration, software development and application deployment.

As we have shown in the results of the tests, the worker is almost autonomous. It can stay up and running for
a long period while it is being charged just by the solar panel. As future improvements, we can use machine
learning to create a model that optimizes the battery consumption. Also, we can use weather forecast services
to predict unexpected weather changes. 